%%%\section{Introduction}

Why study effects of RF antenna on tokamak edge plasmas.

The physics of magnetically confined plasmas in fusion experiments spans a wide range of
timescales, ranging from the very rapid oscillations associated with electron
Langmuir waves and gyromotion to the much longer timescales associated with
transport in the plasma edge and scrape-off layer (SOL).  Although various research efforts
within the plasma physics community (RF, MHD, turbulence/transport, etc.) can
loosely be identified with various timescales within this range, it
is clear that physics on one timescale may significantly affect the physics on another
(e.g. increased radial core transport arising from the growth and saturation of MHD
instabilities such as neoclassical tearing modes).  In this paper, we consider one such
phenomenon -- namely, the generation of ponderomotive forces by fast-timescale RF waves
(launched from antennas in the plasma edge with the objective of heating the core plasma)
and the effect such forces have on transport-timescale transport behavior in the plasma edge/SOL.
For the large RF power fluxes on the order of those used in larger fusion devices such as NSTX,
Alcator C-Mod, or ITER ($\sim$ 1 MW/m$^{2}$), ponderomotive forces may become large enough
to significantly alter the edge transport, and the ensuing changes to plasma density (particularly
in regions immediately around the antenna) also alter the coupling of RF wave energy to the reactor core.

\textcolor{red}{Needs some references.}

