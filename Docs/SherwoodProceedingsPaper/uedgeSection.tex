(LLNL) includes description of UEDGE and the equations it solves.




UEDGE is a two-dimensional fluid transport code for collisional
boundary plasmas plasmas \cite{}. The plasma physics model in UEDGE is
based on Braginskii \cite{}, with the addition of ad hoc anomalous or
turbulence-enhanced transport coefficients for the direction across
the magnetic field; transport along the magnetic field is taken as
classical with flux limits. For neutral gas, UEDGE includes a
collisional fluid model, based on the assumptions that charge-exchange
(CX) collisions are strong. Besides, UEDGE includes some basic models
for plasma-material interactions (PMI) to describe material surface
sputtering etc.

The full UEDGE model includes the following time-evolution equations:\\
(i) plasma density, also impurity ion and neutrals densities \\
(ii) plasma, also impurity ions, parallel momenta \\
(vi) neutral hydrogen parallel momentum \\
(iii) electron temperature \\
(iv) ion (including impurities and neutrals) temperature \\
(v) electric potential

The equations are discretied on a toroidally symmetric 2D
finite-volume grid and integrated in time, typically to a steady
state.

UEDGE has had many dozens of applications since the early 1990s,
including: modeling of divertor experiments (DIII-D, C-Mod, NSTX, and
others), modeling fusion reactors and future tokamaks (ITER, ARIES,
FDF, FNSF, SPARC), and other physics studies \cite{}.

To include physics outside of its model, UEDGE has been coupled to other
codes, e.g., for including kinetric neutral transport \cite{}, kinetic
impurity transport \cite{}, radiation transport \cite{}, plasma
turbulence \cite{}, dust transport \cite{}, transport in material
walls \cite{}. In the present report, we describe coupling of UEDGE
with RF wave physics code Vorpal \cite{}.


\subsection{RF field effects on tokamak edge plasma}

The RF antenna field in boundary plasma is a source of energy, due to
absorbtion of wave energy, and a source of momentum due to the
ponderomotive forces that RF wave exerts on charged particles.

\subsubsection{RF power source}

Normally only a small fraction, on the order of 1$\%$, of the RF
antenna power is absorbed in the SOL (why?). For the RF antenna power
flux and the absorbed power fraction $f_{abs}$, one has the RF power
source in the SOL

\beq
P_{RF} = f_{abs} q_{RF} / \lambda_{sol},
\eeq

where $\lambda_{sol}$ is the SOL width.

Similarly, the power source due to the tokamak exhaust heat flux is

\beq
P_{ex} = q_{ex}	/ \lambda_{sol},
\eeq

where the exhaust heat flux is

\beq
q_{ex} = P_{core} / A_{LCFS}
\eeq

Using parameters of a typical tokamak, $P_{core}$=10 MW, $A_{LCFS}$=10
m$^2$, $q_{RF}$=1 MW/m$^2$, one can conclude that the RF power source
in the SOL is insignificant,

\beq
P_{RF}/P_{ex} = f_{abs} \sim 0.01\ll 1
\eeq


\subsubsection{RF source of perpendicular momentum}

The momentum source from the RF antenna field enters boundary plasma
via ponderomotive (PM) forces. The PM forces perpendicular to the
magnetic field give rise to plasma drifts, which are not entirely
ambipolar, and this leads to perturbing the electric potential
distribution in the plasma, which may have various effects on
it. However, our calculations demonstrate that this effect is very
small for realistic tokamak plasma parameters.

On the other hand, parallel PM forces from the RF antenna, for
realistic tokamak parameters, turn out to produce quite significant
effects. This is consistent with experimental observations showing RF
antenna perturbing tokamak edge plasmas; however there are other
mechanisms (e.g., the RF sheath \cite{} which is beyond the present
study) that may play an important and perhaps even dominant role
there.
