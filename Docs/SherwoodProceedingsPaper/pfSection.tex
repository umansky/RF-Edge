To derive an expression for the ponderomotive force, 
we begin by writing Maxwell's equations together with the plasma kinetic equation;
\begin{eqnarray}
\nabla \cdot {\bf B} & = & 0 \\
{\partial {\bf B} \over \partial t} & = & - \nabla \times {\bf E} \\
\nabla \cdot {\bf E} & = & {1 \over \epsilon_{0}} \sum_{s} \int q_{s} f_{s}~d^{3}{\bf v} \\
{\partial {\bf E} \over \partial t} & = & c^{2} \nabla \times {\bf B} - {1 \over \epsilon_{0}} \sum_{s} 
\int q_{s} {\bf v} f_{s}~d^{3}{\bf v} \\
{\partial f_{s} \over \partial t} & + & {\bf v} \cdot \nabla f_{s} + {q_{s} \over m_{s}}
[{\bf E} + {\bf v} \times {\bf B}] \cdot {\partial f_{s} \over \partial {\bf v}} = 
\sum_{b} C_{s,b} + S_{s} ~.~
\end{eqnarray} 

Here, ${\bf E}={\bf E}({\bf x},t)$ and ${\bf B}={\bf B}({\bf x},t)$ are the electric and magnetic fields;
$f_{s}=f_{s}({\bf x},{\bf v},t)$ is the distribution function of species $s$ whose members have charge
$q_{s}$ and mass $m_{s}$.  The quantities $({\bf x}, {\bf v}, t)$ are position, velocity, and time coordinates
respectively.
In addition, $c$ is the speed of light and $\epsilon_{0}$ the vacuum permittivity.  We represent
collisional processes which species $s$ undergoes with an arbitrary species $b$ (where $b$ may represent
either $s$ itself, or other charged or neutral species) via a generic collison term $C_{s,b}$, and we likewise
represent source or sink terms for species $s$ with the generic term $S_{s}$.  For the moment the
specific form of these source and collisional terms is immaterial.

As alluded to in the introduction, it is of interest to separate the Maxwell and charged-species 
kinetic equations (and fluid moments of the latter) into slow- and fast-timescale parts; the latter
will represent the RF wave physics while the former will be associated with edge/SOL transport. 
This separation will 
ultimately lead to a focus on the pressure term in the fluid momentum equation, and on terms which 
are the product of two fast time-scale quantities. Such 
products have both slow and fast time-scale behavior, as illustrated in the classic trigonometric 
identity, $\sin^{2}(x) = 1⁄2-1⁄2 \cos(2x)$; here the product of $\sin(x)$ with itself generates a 
doubled-frequency oscillatory part and a non-oscillatory part, the latter of which is of primary 
interest here. One should note that the phasing of the two items in the product matters, as for 
example, $\sin(x) \cos(x) = 1⁄2 \sin(2x)$, which has no non-oscillatory part. Thus, a product of 
two different quantities, such as a charge density times an electric field, may or may not have a 
non-oscillatory part, but a product of a quantity with itself will most definitely have a 
non-oscillatory part.  Conceptually, such non-oscillatory components represent physics on the slower
(transport) timescale, even though they arise from products of the fast, oscillatory terms associated 
with RF wave physics.

To separate our equations into fast and slow timescales, we will define complementary, idempotent low-pass
and high-pass filters, designated as $\langle \ldots \rangle$ and $\{ \ldots \}$ respectively.
We will also use a $0$ subscript to denote low-pass-filtered single quantities and a $1$ subscript to
denote high-pass-filtered single quantities.  Accordingly, for an arbitrary function of time (which may
also depend on position and possibly velocity) we may write
\begin{eqnarray}
A(t) & = & \langle A \rangle + \{ A \} \nonumber \\
     & \equiv & A_{0} + A_{1} 
\end{eqnarray}
for operators which satisfy
\begin{eqnarray}
\langle A_{0} \rangle = A_{0} ~~;&~~ \{ A_{1} \} = A_{1} ~~&~~ \mbox{(idempotent)} \\
\langle A_{1} \rangle = 0 ~~;&~~ \{ A_{0} \} = 0 ~~&~~ \mbox{(complementary)}
\end{eqnarray} 
For products of arbitrary functions of time, we stipulate the following relations:
\begin{itemize}
\item Low-frequency (zero-subscripted) quantities can pass in and out of both the $\langle \ldots \rangle$ and $\{ \ldots \}$ operators, so that 
\begin{eqnarray}
\langle \langle A \rangle B \rangle & = & \langle A \rangle \langle B \rangle = A_{0} B_{0} ~,\\
\{ \langle A \rangle B \} & = & \langle A \rangle \{ B \} = A_{0} B_{1} ~, \\
\langle A B \rangle & = & \langle A_{0} B_{0} + A_{1} B_{0} + A_{0} B_{1} + A_{1} B_{1} \rangle \nonumber \\ 
& = & \langle A \rangle \langle B \rangle + \langle \{A\} \{B\} \rangle \nonumber \\
& = & A_{0} B_{0} + \langle A_{1} B_{1} \rangle ~.
\end{eqnarray}
\item The low-frequency average of a constant is unity, so that 
\begin{equation}
\langle A_{0} \rangle = A_{0} \langle 1 \rangle = A_{0} ~.
\end{equation}
\item The high-frequency average of a constant is zero, so that
\begin{equation}
\{ A_{0} \} = A_{0} \{ 1 \} = 0 ~.
\end{equation}
\end{itemize}
No assumption will be made on the relative magnitudes of any quantities $A_{0}$ and $A_{1}$
at this point -- this is spectral decomposition, not linearization as the $1$ subscript
conventionally suggests.

Common fluid moments associated with the distribution function $f_{s}({\bf x},{\bf v},t)$, and
which we will spectrally decompose in the manner indicated, include the following:
\begin{eqnarray}
n_{s} = & \int f_{s}~d^{3}{\bf v} & \mbox{~(density)} \\
\rho_{s} = & \int q_{s} f_{s}~d^{3}{\bf v} & \mbox{~(charge density)} \\
{\bf V}_{s} = & (1/n_{s}) \int f_{s} {\bf v}~d^{3}{\bf v} & \mbox{~(fluid velocity)} \\
{\bf J}_{s} = & \int q_{s} f_{s} {\bf v}~d^{3}{\bf v} & \mbox{~(current density)} \\
{\mathbf p}_{s} = & \int m_{s} f_{s} {\bf v} {\bf v}~d^{3}{\bf v} & 
\mbox{~(lab-frame pressure)} \\
{\mathbf P}_{s} = & \int m_{s} f_{s} ({\bf v} - {\bf V}_{s})({\bf v} - {\bf V}_{s})~d^{3}{\bf v} & 
\mbox{~(fluid-frame pressure) ~~~~~} 
\end{eqnarray}
where all quantities on the left vary only with $({\bf x},t)$. 

Maxwell's equations, when velocity integrals are expressed in terms of the fluid
moments, can be seen to be linear and timescale-separable in $\rho_{s}$ and ${\bf J}_{s}$ 
[i.e. the divergence of slow (fast) electric fields arises from slow (fast) charge 
densities, without any mixing of timescales via products or quotients of terms that vary on both fast and 
slow timescales, such that the relations

\begin{equation}
\nabla \cdot {\bf E}_{0} = {1 \over \epsilon_{0}} \sum_{s} \int q_{s} f_{0s}~d^{3}{\bf v}
= {1 \over \epsilon_{0}} \sum_{s} \rho_{0s}
\end{equation}
and
\begin{equation}
\nabla \cdot {\bf E}_{1} = {1 \over \epsilon_{0}} \sum_{s} \int q_{s} f_{1s}~d^{3}{\bf v}
= {1 \over \epsilon_{0}} \sum_{s} \rho_{1s}
\end{equation}
both independently hold].  The charge-weighted velocity integral of the kinetic equation yields a
continuity equation
\begin{equation}
{\partial \rho_{s} \over \partial t} + \nabla \cdot {\bf J}_{s} = \mbox{~charge sources and sinks}
\end{equation}
which can in principle also easily separate into fast and slow components.
This timescale-separability offers great advantage for the Maxwell and continuity equations,
as equations relevant to each timescale can be considered separately.  However, this 
simplicity is not preserved by higher-order velocity moments of the kinetic equation.
In particular, both the fluid-frame pressure (which is related to the 
species fluid velocity ${\bf V}_{s} = {\bf J}_{s}/\rho_{s}$, a ratio of quantities that vary on both
fast and slow timescales) and terms associated with the Lorentz force (products of charge/current
densities and electromagnetic fields) are not easily separable into purely ``fast'' and ``slow''
components.  We have, for the fluid momentum equation,

\begin{eqnarray}
{\partial {\bf J}_{s} \over \partial t} + \nabla \cdot \left( {{\bf J}_{s} {\bf J}_{s} \over \rho_{s}} \right)
+ {q_{s} \over m_{s}} \nabla \cdot {\mathbf P}_{s} = {q_{s} \over m_{s}} [\rho_{s} {\bf E} + {\bf J}_{s}
\times {\bf B}] \nonumber \\ 
{} + \mbox{~current sources and sinks}
\end{eqnarray}
which can also be written in terms of the species fluid velocity as

\begin{eqnarray}
{m_{s} \over q_{s}} \rho_{s} \left[ {\partial {\bf V}_{s} \over \partial t} +
({\bf V}_{s} \cdot \nabla) {\bf V}_{s} \right] + \nabla \cdot {\mathbf P}_{s} =
\rho_{s}[{\bf E} + {\bf V}_{s} \times {\bf B}] \nonumber \\ 
{} + \mbox{~momentum sources and sinks}
\end{eqnarray} 

${\bf V}_{s}$ and ${\bf J}_{s}$ are two of many interrelated field quantities in the fluid system of 
equations, and all of these quantities participate in products of two or more fields, so where best to 
initiate the analysis of time-scale separation, e.g. at Vs, or at Js, is not particularly evident, from 
a fluid point-of-view. However, Maxwell’s Equation, and the fluid Continuity Equation are linear in $\rho_{s}$, 
${\bf J}_{s}$, ${\bf E}$ and ${\bf B}$, so time-scale separation in terms of these quantities is pure, 
e.g., in these equations Jslow interacts with only rhoslow, Eslow, and Bslow, and not with rhofast, 
Efast and Bfast. This provides a fairly strong incentive to initiate the time-scale separation 
analysis at Js, rather than Vs, which is what we will do in this write-up. This means that the 
product terms only occur in the fluid moment equations beyond the Continuity Equation, the first of 
which is the momentum equation.



