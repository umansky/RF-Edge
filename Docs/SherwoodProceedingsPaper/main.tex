%\documentclass[%
% aip,
% jmp,
% bmf,
% sd,
% rsi,
% amsmath,amssymb,
%preprint,%
% reprint,%
%author-year,%
%author-numerical,%
% Conference Proceedings
%]{revtex4-1}


\documentclass[aip,preprint]{revtex4}


\usepackage{graphicx}% Include figure files
\usepackage{dcolumn}% Align table columns on decimal point
\usepackage{bm}% bold math
%\usepackage[mathlines]{lineno}% Enable numbering of text and display math
%\linenumbers\relax % Commence numbering lines

\usepackage[utf8]{inputenc}
\usepackage[T1]{fontenc}
\usepackage{mathptmx}
\usepackage{etoolbox}
\usepackage{color}

%% Apr 2021: AIP requests that the corresponding 
%% email to be moved after the affiliations
\makeatletter
\def\@email#1#2{%
 \endgroup
 \patchcmd{\titleblock@produce}
  {\frontmatter@RRAPformat}
  {\frontmatter@RRAPformat{\produce@RRAP{*#1\href{mailto:#2}{#2}}}\frontmatter@RRAPformat}
  {}{}
}%
\makeatother

\def\beq{\begin{equation}}
\def\eeq{\end{equation}}
\def\beqar{\begin{eqnarray}}
\def\eeqar{\end{eqnarray}}
\def\nn{\nonumber}


\begin{document}

\preprint{AIP/123-QED}

\title[Modeling of RF-induced ponderomotive effects on tokamak boundary transport]
      {Modeling of RF-induced ponderomotive effects on tokamak boundary transport} 
\author{Thomas G. Jenkins}
\homepage{https://nucleus.txcorp.com/~tgjenkins}
\author{David N. Smithe}
\affiliation{Tech-X Corporation, 5621 Arapahoe Avenue Suite A, Boulder CO 80303, USA}
\author{Maxim V. Umansky}
\author{Andris M. Dimits}
\author{Thomas D. Rognlien}
\affiliation{Lawrence Livermore National Laboratory, Livermore CA 94550, USA}
\email{tgjenkins@txcorp.com}
 
\date{\today}

\begin{abstract}
Large-amplitude electromagnetic fields driven by RF antennas in the tokamak plasma edge induce perturbed 
charge and current densities on fast RF timescales as power is injected to heat the core plasma. However, 
these fields, charges, and currents also give rise to slow (transport-timescale) ponderomotive effects, since 
the time-average of products of fast quantities is nonzero on these longer timescales. Together with the 
conventional $\nabla(|{\bf E}|^{2})$ ponderomotive force associated with nonuniform EM energy density, 
additional force terms dependent on density gradients, species charge signs, collisionality, and incident 
wave polarization arise, and these terms introduce new vorticity, energy, and parallel momentum sources 
to the edge and SOL transport.  We have coupled the Vorpal (FDTD EM+plasma solver) and UEDGE (2D 
edge plasma transport) codes in a manner enabling numerical study of these ponderomotive terms and 
their effects on edge/SOL transport. In NSTX-adjacent scenarios with experimentally realistic plasma 
profiles and antenna parameters, we observe that the ponderomotive contribution to parallel electron 
momentum is significant (comparable to or larger than other edge transport processes) for representative 
RF input power fluxes. We demonstrate that this parallel momentum source drives the transport of density 
away from the region immediately in front of the RF antenna. Further, as the density is reduced, we show 
that the polarization, propagation, and absorption of incident RF waves is accordingly modified, often in 
detrimental (for antenna efficiency and core power coupling) and potentially self-reinforcing ways.  
Because ponderomotive effects scale in magnitude with the antenna input power flux, they become 
increasingly relevant for large, high-RF-power experiments such as ITER (input power flux 
$\sim 1~\mbox{MW/m}^2$), and we consider the implications of these results for ITER antenna operation.
\end{abstract}

\maketitle

\section{\label{sec:intro}Introduction}
%%%\section{Introduction}

Why study effects of RF antenna on tokamak edge plasmas.


\section{\label{sec:ponderomotive}Ponderomotive Force}
To derive an expression for the ponderomotive force, 
we begin by writing Maxwell's equations together with the plasma kinetic equation;
\begin{eqnarray}
\nabla \cdot {\bf B} & = & 0 \\
{\partial {\bf B} \over \partial t} & = & - \nabla \times {\bf E} \\
\nabla \cdot {\bf E} & = & {1 \over \epsilon_{0}} \sum_{s} \int q_{s} f_{s}~d^{3}{\bf v} \\
{\partial {\bf E} \over \partial t} & = & c^{2} \nabla \times {\bf B} - {1 \over \epsilon_{0}} \sum_{s} 
\int q_{s} {\bf v} f_{s}~d^{3}{\bf v} \\
{\partial f_{s} \over \partial t} & + & {\bf v} \cdot \nabla f_{s} + {q_{s} \over m_{s}}
[{\bf E} + {\bf v} \times {\bf B}] \cdot {\partial f_{s} \over \partial {\bf v}} = 
\sum_{b} C_{s,b} + S_{s} ~.~
\end{eqnarray} 

Here, ${\bf E}={\bf E}({\bf x},t)$ and ${\bf B}={\bf B}({\bf x},t)$ are the electric and magnetic fields;
$f_{s}=f_{s}({\bf x},{\bf v},t)$ is the distribution function of species $s$ whose members have charge
$q_{s}$ and mass $m_{s}$.  The quantities $({\bf x}, {\bf v}, t)$ are position, velocity, and time coordinates
respectively.
In addition, $c$ is the speed of light and $\epsilon_{0}$ the vacuum permittivity.  We represent
collisional processes which species $s$ undergoes with an arbitrary species $b$ (where $b$ may represent
either $s$ itself, or other charged or neutral species) via a generic collison term $C_{s,b}$, and we likewise
represent source or sink terms for species $s$ with the generic term $S_{s}$.  For the moment the
specific form of these source and collisional terms is immaterial.

As alluded to in the introduction, it is of interest to separate the Maxwell and charged-species 
kinetic equations (and fluid moments of the latter) into slow- and fast-timescale parts; the latter
will represent the RF wave physics while the former will be associated with edge/SOL transport. 
This separation will 
ultimately lead to a focus on the pressure term in the fluid momentum equation, and on terms which 
are the product of two fast time-scale quantities. Such 
products have both slow and fast time-scale behavior, as illustrated in the classic trigonometric 
identity, $\sin^{2}(\omega t) = 1⁄2-1⁄2 \cos(2\omega t)$; here the product of $\sin(\omega t)$ with 
itself generates a 
doubled-frequency oscillatory part and a non-oscillatory part, the latter of which is of primary 
interest here. One should note that the phasing of the two items in the product matters, as for 
example, $\sin(\omega t) \cos(\omega t) = 1⁄2 \sin(2\omega t)$, which has no non-oscillatory part. Thus, a product of 
two different quantities, such as a charge density times an electric field, may or may not have a 
non-oscillatory part, but a product of a quantity with itself will most definitely have a 
non-oscillatory part.  Conceptually, such non-oscillatory components represent physics on the slower
(transport) timescale, even though they arise from products of the fast, oscillatory terms associated 
with RF wave physics.

To separate our equations into fast and slow timescales, we will define complementary, idempotent low-pass
and high-pass filters, designated as $\langle \ldots \rangle$ and $\{ \ldots \}$ respectively.
We will also use a ``$0$'' subscript to denote low-pass-filtered single quantities and a ``$1$'' subscript to
denote high-pass-filtered single quantities.  Accordingly, for an arbitrary function of time (which may
also depend on position and possibly velocity) we may write the spectral decomposition
\begin{eqnarray}
A(t) & = & \langle A \rangle + \{ A \} \nonumber \\
     & \equiv & A_{0} + A_{1} 
\end{eqnarray}
which separates $A$ into ``slow'' and ``fast'' timescales.  The spectral operators satisfy
\begin{eqnarray}
\langle A_{0} \rangle = A_{0} ~~;&~~ \{ A_{1} \} = A_{1} ~~&~~ \mbox{(idempotent)} \label{i1} \\
\langle A_{1} \rangle = 0 ~~;&~~ \{ A_{0} \} = 0 ~~&~~ \mbox{(complementary)} \label{i2}
\end{eqnarray} 
We will further assume that practically speaking, the highest representative frequency of the ``slow''
timescale is well-separated from the lowest representative frequency of the ``fast'' timescale, such
that a broad range of null frequencies exists between the two timescales.  Thus,
any sidebands produced by products of two physical variables fall unambiguously within these ``slow'' or ``fast''
ranges;
\begin{eqnarray}
\langle A_{0} B_{0} \rangle = A_{0} B_{0} ~;~&
\langle A_{0} B_{1} \rangle = 0 ~;~&
\langle A_{1} B_{1} \rangle \neq 0 \\
\{ A_{0} B_{0} \} = 0 ~;~&
\{ A_{0} B_{1} \} = A_{0} B_{1} ~;~&
\{ A_{1} B_{1} \} \neq 0
\end{eqnarray}
In essence, this means that (a) low-frequency (zero-subscripted) quantities can pass in and out of
both spectral averaging operators; (b) that the high-frequency average of a constant is zero; and
(c) that the low-frequency average of a constant is that same constant.
The fast-time and slow-time averages of products of physical variables then take the form 
\begin{eqnarray}
\langle A B \rangle & = & A_{0} B_{0} + \langle A_{1} B_{1} \rangle \label{slowprod} \\
\{ A B \} & = & A_{0} B_{1} + A_{1} B_{0} + \{ A_{1} B_{1} \} \label{fastprod}
\end{eqnarray}
No assumption will be made on the relative magnitudes of any quantities $A_{0}$ and $A_{1}$
at this point -- although the $1$ subscript is conventionally used to denote linearization
(implying that $A_{1}/A_{0} \ll 1$), that convention is not used here.  We reiterate that 
$A_{0}$ refers to the slow timescale, low-frequency behavior of physical variable $A$,  while 
$A_{1}$ refers to its fast timescale, high-frequency behavior.

Common fluid moments associated with the distribution function $f_{s}({\bf x},{\bf v},t)$,
which we will spectrally decompose in the manner indicated, include the following:
\begin{eqnarray}
n_{s} = & \int f_{s}~d^{3}{\bf v} & \mbox{~(density)} \label{densitymoment} \\
\rho_{s} = & \int q_{s} f_{s}~d^{3}{\bf v} & \mbox{~(charge density)} \\
{\bf V}_{s} = & (1/n_{s}) \int f_{s} {\bf v}~d^{3}{\bf v} & \mbox{~(fluid velocity)} \\
{\bf J}_{s} = & \int q_{s} f_{s} {\bf v}~d^{3}{\bf v} & \mbox{~(current density)} \\
{\mathbf p}_{s} = & \int m_{s} f_{s} {\bf v} {\bf v}~d^{3}{\bf v} & 
\mbox{~(lab-frame pressure)} \\
{\mathbf P}_{s} = & \int m_{s} f_{s} ({\bf v} - {\bf V}_{s})({\bf v} - {\bf V}_{s})~d^{3}{\bf v} & 
\mbox{~(fluid-frame pressure) ~~~~~} \label{ffpressuremoment} 
\end{eqnarray}
where all quantities on the left vary only with $({\bf x},t)$.  Moments which depend
on the distribution function but not on other multiple-timescale quantities will satisfy simple 
rules for spectral decomposition, e.g. since $f_{s} = f_{0s} + f_{1s}$ we also have $n_{s} = n_{0s} + n_{1s}, 
\rho_{s} = \rho_{0s} + \rho_{1s}, {\bf J}_{s} = {\bf J}_{0s} + {\bf J}_{1s}$, and  
${\mathbf p}_{s} = {\mathbf p}_{0s} + {\mathbf p}_{1s}$.  However, moments with dependence on
multiple variables [e.g. ${\bf V}_{s}(n_{s},f_{s})$ or ${\mathbf P}_{s}(f_{s},{\bf V}_{s})$]
will demand more complicated spectral decompositions which will be discussed shortly.

Maxwell's equations, when velocity integrals are expressed in terms of the fluid
moments, are timescale-separable in the fast and slow components of $\rho_{s}$, ${\bf J}_{s}$ and
the electromagnetic fields.  By ``timescale-separable'' we mean that 
the divergence of slow (fast) electric fields arises from slow (fast) charge 
densities, without any mixing of timescales via products or quotients of terms that vary on both fast and 
slow timescales, such that the relations

\begin{equation}
\nabla \cdot {\bf E}_{0} = {1 \over \epsilon_{0}} \sum_{s} \int q_{s} f_{0s}~d^{3}{\bf v}
= {1 \over \epsilon_{0}} \sum_{s} \rho_{0s}
\end{equation}
and
\begin{equation}
\nabla \cdot {\bf E}_{1} = {1 \over \epsilon_{0}} \sum_{s} \int q_{s} f_{1s}~d^{3}{\bf v}
= {1 \over \epsilon_{0}} \sum_{s} \rho_{1s}
\end{equation}
both independently hold.  The charge-weighted velocity integral of the kinetic equation yields a
continuity equation which is likewise timescale-separable, though
sources and sinks relevant to the fast or slow timescales introduce some subtleties -- we
may independently write
\begin{equation}
{\partial \rho_{0s} \over \partial t} + \nabla \cdot {\bf J}_{0s} = \langle \mbox{~charge sources and sinks} \rangle
\end{equation}
and
\begin{equation}
{\partial \rho_{1s} \over \partial t} + \nabla \cdot {\bf J}_{1s} = \{ \mbox{~charge sources and sinks} \}
\end{equation}
and separately consider these relations even without a specified form for the source and sink
terms.  
Unfortunately, such timescale-separability is not preserved by higher-order
velocity moments of the kinetic equation.
In particular, neither the species fluid velocity ${\bf V}_{s} = {\bf J}_{s}/\rho_{s}$ (a ratio of 
quantities that vary on both fast and slow timescales), the fluid-frame species pressure (which 
is computed relative to ${\bf V}_{s}$), 
nor terms associated with the Lorentz force (products of charge/current
densities and electromagnetic fields) are easily separable into purely ``fast'' and ``slow''
components.  Considering the fluid momentum equation in its simplest form,
\begin{eqnarray}
{\partial {\bf J}_{s} \over \partial t} 
%+ \nabla \cdot \left( {{\bf J}_{s} {\bf J}_{s} \over \rho_{s}} \right)
+ {q_{s} \over m_{s}} \nabla \cdot {\mathbf p}_{s} = {q_{s} \over m_{s}} [\rho_{s} {\bf E} + {\bf J}_{s}
\times {\bf B}] + \mbox{~current sources and sinks}
\label{fluidmomentum}
\end{eqnarray}
we see that although the left-hand side is timescale-separable into purely slow or fast components, the
Lorentz-force terms on the right-hand side are products and will thus have quadratic terms in
both their fast- and slow-timescale averages [as in Eqs.~(\ref{slowprod} - \ref{fastprod}) above].  
Further, when Eq.~(\ref{fluidmomentum}) is rewritten in terms of the species flow velocity and fluid-frame
pressure (quantities more relevant to the study of edge and scrape-off layer transport), 
additional product terms arise; we have
\begin{eqnarray}
{m_{s} \over q_{s}} \rho_{s} \left[ {\partial {\bf V}_{s} \over \partial t} +
({\bf V}_{s} \cdot \nabla) {\bf V}_{s} \right] + \nabla \cdot {\mathbf P}_{s} =
\rho_{s}[{\bf E} + {\bf V}_{s} \times {\bf B}] + \mbox{~momentum sources and sinks}
\end{eqnarray} 
in which neither ${\bf V}_{s}$ nor ${\mathbf P}_{s}$ are cleanly timescale-separable.
We must then carefully consider what form this momentum equation will take on the fast and slow timescales.

Fundamentally, while both ${\bf J}_{s} = {\bf J}_{0s} + {\bf J}_{1s}$ and $\rho_{s} = \rho_{0s}
+ \rho_{1s}$ separate cleanly into fast and slow components (arising from their construction
as velocity moments of a separable distribution function $f_{s} = f_{0s} + f_{1s}$), the velocity
${\bf V}_{s} = {\bf J}_{s}/\rho_{s}$ does not.  Nevertheless, such components can be 
constructed by expanding ${\bf V}_{s}$ in the form
\begin{equation}
{\bf V}_{s} = {{\bf J}_{0s} + {\bf J}_{1s} \over \rho_{0s} + \rho_{1s}} =
{{\bf J}_{0s} \over \rho_{0s}} \left( 1 - {\rho_{1s} \over \rho_{s}} \right) +
{{\bf J}_{1s} \over \rho_{s}} =
{{\bf J}_{0s} \over \rho_{0s}} + {{\bf J}_{1s} - {\bf J}_{0s} \rho_{1s}/\rho_{0s} \over \rho_{s}} 
\end{equation}
Defining an unambiguously slow-timescale quantity ${\bf V}_{0s} \equiv {\bf J}_{0s}/\rho_{0s}$
for convenience, we then have
\begin{equation}
{\bf V}_{s} = {\bf V}_{0s} + {{\bf J}_{1s} - {\bf V}_{0s} \rho_{1s} \over \rho_{s}} 
= {\bf V}_{0s} + {{\bf J}_{1s} - {\bf V}_{0s} \rho_{1s} \over \rho_{0s}}
\left(1 - {\rho_{1s} \over \rho_{s}} \right)
\end{equation} 
Then, similarly defining an unambiguously fast-timescale quantity ${\bf V}_{1s} \equiv
({\bf J}_{1s} - {\bf V}_{0s} \rho_{1s})/\rho_{0s}$ for convenience, we may write 
\begin{equation}
{\bf V}_{s} = {\bf V}_{0s} + {\bf V}_{1s} - {\rho_{1s} {\bf V}_{1s} \over \rho_{s}}
\end{equation} 
whose last term is at least quadratic in fast quantities.  We have defined our new terms
to satisfy ${\bf J}_{0s} = \rho_{0s} {\bf V}_{0s}$ and ${\bf J}_{1s} = \rho_{0s} {\bf V}_{1s}
+ \rho_{1s} {\bf V}_{0s}$, consistent with the rules for fast-time and slow-time averages
defined in Eq.~(\ref{i1} - \ref{i2}).  Nevertheless, applying such averages
to the fluid velocity ${\bf V}_{s}$ also yields quadratic terms,
\begin{eqnarray}
\langle {\bf V}_{s} \rangle & = & {\bf V}_{0s} - \left\langle {\rho_{1s} {\bf V}_{1s} \over \rho_{s}} \right\rangle \label{V0eqn} \\
\{ {\bf V}_{s} \} & = & {\bf V}_{1s} - \left\{ {\rho_{1s} {\bf V}_{1s} \over \rho_{s}} \right\} \label{V1eqn}
\end{eqnarray}
on both fast and slow timescales.  Unlike ${\bf J}_{0s}$ and ${\bf J}_{1s}$, which respectively
describe the entirety of species
currents on slow and fast timescales, the velocities ${\bf V}_{0s}$ and ${\bf V}_{1s}$ describe only a 
portion of the species flow velocity on these timescales.  Nevertheless, such notation for
the velocity simplifies the deriviation of the ponderomotive force terms tremendously.

Slow-timescale and fast-timescale versions of the fluid-frame pressure tensor can also be constructed,
in the same manner as is done in Eqs.~(\ref{V0eqn} - \ref{V1eqn}) for the species velocity.
Expanding the relation ${\mathbf P}_{s} = {\mathbf p}_{s} - (m_{s}/q_{s}) \rho_{s} {\mathbf V}_{s}
{\mathbf V}_{s}$ [which can be obtained by manipulating Eqs.~(\ref{densitymoment} - \ref{ffpressuremoment})],
it can be shown that
\begin{equation}
\langle {\mathbf P}_{s} \rangle = {\mathbf P}_{0s} 
- {m_{s} \over q_{s}} \rho_{0s}^{2} \left\langle {{\bf V}_{1s} {\bf V}_{1s} \over \rho_{s}} \right\rangle
\end{equation}
and
\begin{equation}
\left\{ {\mathbf P}_{s} \right\} = {\mathbf P}_{1s}
- {m_{s} \over q_{s}} \rho_{0s}^{2} \left\{ {{\bf V}_{1s} {\bf V}_{1s} \over \rho_{s}} \right\}
\end{equation}
wherein unambiguously slow-timescale and fast-timescale fluid-frame pressure quantities have been defined,
\begin{eqnarray}
{\mathbf P}_{0s} & \equiv & {\mathbf p}_{0s} - {m_{s} \over q_{s}} \rho_{0s} {\bf V}_{0s} {\bf V}_{0s} \\
{\mathbf P}_{1s} & \equiv & {\mathbf p}_{1s} - {m_{s} \over q_{s}} [
\rho_{1s} {\bf V}_{0s} {\bf V}_{0s} + \rho_{0s} {\bf V}_{1s} {\bf V}_{0s} + \rho_{0s} {\bf V}_{0s} {\bf V}_{1s}]
\end{eqnarray}
Analogously to ${\bf V}_{0s}$ and ${\bf V}_{1s}$, these pressure tensors describe only a portion of the full
species pressures on the slow and fast timescales, which can be written respectively as
\begin{eqnarray}
\langle {\mathbf P}_{s} \rangle & = & {\mathbf p}_{0s} - {m_{s} \over q_{s}} \langle \rho_{s} {\bf V}_{s} {\bf V}_{s} \rangle \\
\{ {\mathbf P}_{s} \} & = & {\mathbf p}_{1s} - {m_{s} \over q_{s}} \{ \rho_{s} {\bf V}_{s} {\bf V}_{s}
\}
\end{eqnarray}
using the relation between ${\mathbf P}_{s}$ and ${\mathbf p}_{s}$ above.


\section{\label{sec:modelingtools}Modeling Tools}
\subsection{\label{sec:uedge}UEDGE}
(LLNL) includes description of UEDGE and the equations it solves.

\subsection{\label{sec:vorpal}Vorpal}
(Tech-X) includes description of Vorpal and the equations it solves.

\subsection{\label{sec:coupling}Coupling Strategy}
(Tech-X) Describes how UEDGE and Vorpal are coupled in our model.


\section{\label{sec:slabmodel}Numerical Model}
(Tech-X) Short introduction to this section.

\subsection{\label{sec:slab}An NSTX-like Slab Model}
(Tech-X/LLNL) Describes NSTX-like slab model.

\subsection{\label{sec:parallelforce}Parallel Force}
(LLNL) Describes why we think the parallel ponderomotive force is more important than
other physics (vorticity, cross-field forces, etc.) and asserts that we will only be
looking at the parallel forces hereafter.


\section{\label{sec:results}Results}
(Tech-X/LLNL) What we found.


\section{\label{sec:implications}Implications and Future Work}
(Tech-X/LLNL) Discusses the implications of this work and what we will do next.


\begin{acknowledgments}
This work is supported by the U.S. Department of Energy’s Office of Fusion Energy Sciences in 
connection with the SciDAC Center for Integrated Simulation of Fusion Relevant RF Actuators 
(rf-SciDAC), under contracts DE-AC52-07NA27344 (LLNL), FWP-2017-LLNL-SCW1619 (LLNL), and 
DE-SC0018319 (Tech-X).

\end{acknowledgments}

\section*{Data Availability Statement}
The data that support the findings of this study are available from the corresponding author upon reasonable request.

\appendix

\section{Appendixes}

%\nocite{*} % to see all entries in .bib file whether or not they were cited
\bibliography{ponderomotive}% Produces the bibliography via BibTeX.

\end{document}

% sample cites
%\cite{feyn54,witten2001,epr,Berman1983}, 
%\onlinecite{epr,feyn54,Bire82,Berman1983} 
%Can put footnotes into the bibliography\footnote{like this.}
