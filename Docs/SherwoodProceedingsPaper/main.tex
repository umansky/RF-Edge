%\documentclass[%
% aip,
% jmp,
% bmf,
% sd,
% rsi,
% amsmath,amssymb,
%preprint,%
% reprint,%
%author-year,%
%author-numerical,%
% Conference Proceedings
%]{revtex4-1}


\documentclass[aip,preprint]{revtex4}


\usepackage{graphicx}% Include figure files
\usepackage{dcolumn}% Align table columns on decimal point
\usepackage{bm}% bold math
%\usepackage[mathlines]{lineno}% Enable numbering of text and display math
%\linenumbers\relax % Commence numbering lines

\usepackage[utf8]{inputenc}
\usepackage[T1]{fontenc}
\usepackage{mathptmx}
\usepackage{etoolbox}

%% Apr 2021: AIP requests that the corresponding 
%% email to be moved after the affiliations
\makeatletter
\def\@email#1#2{%
 \endgroup
 \patchcmd{\titleblock@produce}
  {\frontmatter@RRAPformat}
  {\frontmatter@RRAPformat{\produce@RRAP{*#1\href{mailto:#2}{#2}}}\frontmatter@RRAPformat}
  {}{}
}%
\makeatother

\def\beq{\begin{equation}}
\def\eeq{\end{equation}}
\def\beqar{\begin{eqnarray}}
\def\eeqar{\end{eqnarray}}
\def\nn{\nonumber}


\begin{document}

\preprint{AIP/123-QED}

\title[Modeling of RF-induced ponderomotive effects on tokamak boundary transport]
      {Modeling of RF-induced ponderomotive effects on tokamak boundary transport} 
\author{Thomas G. Jenkins}
\homepage{https://nucleus.txcorp.com/~tgjenkins}
\author{David N. Smithe}
\affiliation{Tech-X Corporation, 5621 Arapahoe Avenue Suite A, Boulder CO 80303, USA}
\author{Maxim V. Umansky}
\author{Andris M. Dimits}
\author{Thomas D. Rognlien}
\affiliation{Lawrence Livermore National Laboratory, Livermore CA 94550, USA}
\email{tgjenkins@txcorp.com}
 
\date{\today}

\begin{abstract}
Large-amplitude electromagnetic fields driven by RF antennas in the tokamak plasma edge induce perturbed 
charge and current densities on fast RF timescales as power is injected to heat the core plasma. However, 
these fields, charges, and currents also give rise to slow (transport-timescale) ponderomotive effects, since 
the time-average of products of fast quantities is nonzero on these longer timescales. Together with the 
conventional $\nabla(|{\bf E}|^{2})$ ponderomotive force associated with nonuniform EM energy density, 
additional force terms dependent on density gradients, species charge signs, collisionality, and incident 
wave polarization arise, and these terms introduce new vorticity, energy, and parallel momentum sources 
to the edge and SOL transport.  We have coupled the Vorpal (FDTD EM+plasma solver) and UEDGE (2D 
edge plasma transport) codes in a manner enabling numerical study of these ponderomotive terms and 
their effects on edge/SOL transport. In NSTX-adjacent scenarios with experimentally realistic plasma 
profiles and antenna parameters, we observe that the ponderomotive contribution to parallel electron 
momentum is significant (comparable to or larger than other edge transport processes) for representative 
RF input power fluxes. We demonstrate that this parallel momentum source drives the transport of density 
away from the region immediately in front of the RF antenna. Further, as the density is reduced, we show 
that the polarization, propagation, and absorption of incident RF waves is accordingly modified, often in 
detrimental (for antenna efficiency and core power coupling) and potentially self-reinforcing ways.  
Because ponderomotive effects scale in magnitude with the antenna input power flux, they become 
increasingly relevant for large, high-RF-power experiments such as ITER (input power flux 
$\sim 1~\mbox{MW/m}^2$), and we consider the implications of these results for ITER antenna operation.
\end{abstract}

\maketitle

\section{\label{sec:intro}Introduction}
%%%\section{Introduction}

Why study effects of RF antenna on tokamak edge plasmas.


\section{\label{sec:ponderomotive}Ponderomotive Force}
(Tech-X) Includes David's initial derivation.


\section{\label{sec:modelingtools}Modeling Tools}
\subsection{\label{sec:uedge}UEDGE}
(LLNL) includes description of UEDGE and the equations it solves.




UEDGE is a two-dimensional fluid transport code for collisional
boundary plasmas plasmas \cite{}. The plasma physics model in UEDGE is
based on Braginskii \cite{}, with the addition of ad hoc anomalous or
turbulence-enhanced transport coefficients for the direction across
the magnetic field; transport along the magnetic field is taken as
classical with flux limits. For neutral gas, UEDGE includes a
collisional fluid model, based on the assumptions that charge-exchange
(CX) collisions are strong. Besides, UEDGE includes some basic models
for plasma-material interactions (PMI) to describe material surface
sputtering etc.

The full UEDGE model includes the following time-evolution equations:\\
(i) plasma density, also impurity ion and neutrals densities \\
(ii) plasma, also impurity ions, parallel momenta \\
(vi) neutral hydrogen parallel momentum \\
(iii) electron temperature \\
(iv) ion (including impurities and neutrals) temperature \\
(v) electric potential

The equations are discretied on a toroidally symmetric 2D
finite-volume grid and integrated in time, typically to a steady
state.

UEDGE has had many dozens of applications since the early 1990s,
including: modeling of divertor experiments (DIII-D, C-Mod, NSTX, and
others), modeling fusion reactors and future tokamaks (ITER, ARIES,
FDF, FNSF, SPARC), and other physics studies \cite{}.

To include physics outside of its model, UEDGE has been coupled to other
codes, e.g., for including kinetric neutral transport \cite{}, kinetic
impurity transport \cite{}, radiation transport \cite{}, plasma
turbulence \cite{}, dust transport \cite{}, transport in material
walls \cite{}. In the present report, we describe coupling of UEDGE
with RF wave physics code Vorpal \cite{}.


\subsection{RF field effects on tokamak edge plasma}

The RF antenna field in boundary plasma is a source of energy, due to
absorbtion of wave energy, and a source of momentum due to the
ponderomotive forces that RF wave exerts on charged particles.

\subsubsection{RF power source}

Normally only a small fraction, on the order of 1$\%$, of the RF
antenna power is absorbed in the SOL (why?). For the RF antenna power
flux and the absorbed power fraction $f_{abs}$, one has the RF power
source in the SOL

\beq
P_{RF} = f_{abs} q_{RF} / \lambda_{sol},
\eeq

where $\lambda_{sol}$ is the SOL width.

Similarly, the power source due to the tokamak exhaust heat flux is

\beq
P_{ex} = q_{ex}	/ \lambda_{sol},
\eeq

where the exhaust heat flux is

\beq
q_{ex} = P_{core} / A_{LCFS}
\eeq

Using parameters of a typical tokamak, $P_{core}$=10 MW, $A_{LCFS}$=10
m$^2$, $q_{RF}$=1 MW/m$^2$, one can conclude that the RF power source
in the SOL is insignificant,

\beq
P_{RF}/P_{ex} = f_{abs} \sim 0.01\ll 1
\eeq


\subsubsection{RF source of perpendicular momentum}

The momentum source from the RF antenna field enters boundary plasma
via ponderomotive (PM) forces. The PM forces perpendicular to the
magnetic field give rise to plasma drifts, which are not entirely
ambipolar, and this leads to perturbing the electric potential
distribution in the plasma, which may have various effects on
it. However, our calculations demonstrate that this effect is very
small for realistic tokamak plasma parameters.

On the other hand, parallel PM forces from the RF antenna, for
realistic tokamak parameters, turn out to produce quite significant
effects. This is consistent with experimental observations showing RF
antenna perturbing tokamak edge plasmas; however there are other
mechanisms (e.g., the RF sheath \cite{} which is beyond the present
study) that may play an important and perhaps even dominant role
there.

\subsection{\label{sec:vorpal}Vorpal}
(Tech-X) includes description of Vorpal and the equations it solves.

\subsection{\label{sec:coupling}Coupling Strategy}
(Tech-X) Describes how UEDGE and Vorpal are coupled in our model.


\section{\label{sec:slabmodel}Numerical Model}
(Tech-X) Short introduction to this section.

\subsection{\label{sec:slab}An NSTX-like Slab Model}
(Tech-X/LLNL) Describes NSTX-like slab model.

\subsection{\label{sec:parallelforce}Parallel Force}
(LLNL) Describes why we think the parallel ponderomotive force is more important than
other physics (vorticity, cross-field forces, etc.) and asserts that we will only be
looking at the parallel forces hereafter.


\section{\label{sec:results}Results}
(Tech-X/LLNL) What we found.


\section{\label{sec:implications}Implications and Future Work}
(Tech-X/LLNL) Discusses the implications of this work and what we will do next.


\begin{acknowledgments}
This work is supported by the U.S. Department of Energy’s Office of Fusion Energy Sciences in 
connection with the SciDAC Center for Integrated Simulation of Fusion Relevant RF Actuators 
(rf-SciDAC), under contracts DE-AC52-07NA27344 (LLNL), FWP-2017-LLNL-SCW1619 (LLNL), and 
DE-SC0018319 (Tech-X).

\end{acknowledgments}

\section*{Data Availability Statement}
The data that support the findings of this study are available from the corresponding author upon reasonable request.

\appendix

\section{Appendixes}

%\nocite{*} % to see all entries in .bib file whether or not they were cited
\bibliography{ponderomotive}% Produces the bibliography via BibTeX.

\end{document}

% sample cites
%\cite{feyn54,witten2001,epr,Berman1983}, 
%\onlinecite{epr,feyn54,Bire82,Berman1983} 
%Can put footnotes into the bibliography\footnote{like this.}
